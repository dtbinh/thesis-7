% Chapter 1

\chapter{Introduction} % Main chapter title

\label{Introduction} % For referencing the chapter elsewhere, use \ref{Chapter1}

%\lhead{Chapter 1. \emph{Introduction}} % This is for the header on each page - perhaps a shortened title

%-------------------------------------------------------------- --------------------------

The technology surrounding Unmanned Aerial Vehicles (UAVs), and in particular quad-rotor devices, has seen tremendous development in recent years. Likewise, the creative application of this technology has expanded into many contexts. Like many new technologies, the early development of UAVs was mostly in a military context. This is not the case any more. The private sector has taken a huge interest in this technology. There is a wide range of companies contributing to the development of UAV technology from open-source projects like DIY Drones \cite{diydrones:2014:Online} to start-up firms backed by Google, such as Airware \cite{Airware:2014:Online}.  The Federal Aviation Administration in the USA has plans to produce concrete policy regarding the regulation of commercial applications of UAVs by 2015 \cite{faa:2014:Online}. This will sow the seeds for the rapid growth of a multi-billion dollar industry. There are many applications for this technology which have the potential to save lives and collect scientific data that could inform state and federal legislation. Certainly, the range of potential applications will be further diversified as the technology sees more development.

Unmanned Ariel Vehicles are also called by various other names: remotely piloted vehicles (RPVs), remote controlled drones, robot planes, and pilot-less aircraft. Such vehicles are defined as powered, aerial vehicles that do not carry a human operator and can use aerodynamics forces to provide vehicle lift. They can fly autonomously or be piloted remotely, can be expendable or recoverable, and can carry a lethal or nonlethal payload \cite{bone2004unmanned}.


\section{Motivation}

With many private organizations making use of UAVs for a variety of applications, one pervasive engineering problem that still exists in general is that of managing the energy usage. Quad-rotors specifically are plagued by very high energy demand. There are two different kinds of UAVs: fixed wing, which have the ability to soar or glide, and multi-rotor systems, which are entirely thrust-driven. It is the natural instability of multi-rotor UAVs which make them extremely maneuverable, but this comes at the cost of high energy expenditure. This provides the motivation for this thesis - to develop an optimal control technique that optimizes the path-energy of a quad-copter UAV.


\section{Prior Work}

There have been work published on the energy optimization and trajectory planning of fixed wing UAVs. Given the ability to soar and utilize thermal gradients in the atmosphere, it is suggested that fixed wing UAVs have the potential to stay aloft almost permanently \cite{langelaan2007long}, \cite{klesh2009solar}, and \cite{lawrance2009guidance}.  This makes fixed wing UAVs ideal for applications like aerial surveys or surveillance missions. These aircrafts have the major disadvantage that they are dependent upon some kind of launching mechanism, or a runway, for takeoff and landing.

In contrast, rotary wing UAVs have a higher degree of mechanical complexity. These UAVs can take off and land vertically and have the capacity to hover. This makes rotary wing UAVs, such as the quad-copter, more suitable for short range search and rescue missions, facility inspections, and single-target tracking. Since fixed wing and multi-rotor UAVs are fundamentally different in their physical operation, procedures for managing their respective energy usage are also necessarily unique.

In academic contexts, many advances in UAV and specifically quad-rotor research have provided the seeds for growth for this industry. The problem of basic stability and position control is solved in \cite{erginer2007modeling}, \cite{bouabdallah2004pid}, and \cite{Luukkonen}.


The background material for understanding the dynamical model of the quad-rotor as given by the Euler-Lagrange formulation is explained in \cite{marion1995classical} and  \cite{cornelius1970variational}. These references provide detailed derivations and discussions of the Euler-Lagrange equations of motion as well as related topics like Hamiltonian mechanics and the calculus of variations. Also, \cite{cornelius1970variational} provides an in-depth review of the historical context surrounding the development of classical mechanics. The derivation of the quad-rotor dynamical model as well as attitude and position control via PD or PID controllers is discussed in \cite{erginer2007modeling}, \cite{bouabdallah2004pid}, and  \cite{Luukkonen}. These papers provide derivations of both the Euler-Lagrange and Newtonian formulations for the quad-rotor.

Optimal control was born in 1697, when Johann Bernoulli published his solution to the Brachystochrone problem in \cite{JBernoulli} . With the work of Bernoulli, Newton, Leibniz, l'Hopital, and Tschirnhaus, the field of optimal control was clearly defined. This was followed by the works of Euler, Lagrange, and Legendre which led to the fundamental optimization equations, Euler's equation \cite{LEuler}, the Euler-Lagrange formulation, and Legendre's necessary condition for a minimum. W. R. Hamilton then came up with an equivalent to the Euler-Lagrange equation that could be used in deriving control equations. This was known as the control Hamiltonian form of the Euler-Lagrange equations. The next development was from Weierstass, who came up with the fundamental path optimization problem in optimal control theory in the late 19th century. This was followed by the fundamental minimization principle by Pontryagin that allows for solving most optimization problems \cite{Pontry}. Several books on optimal control (\cite{lewis2012optimal} , \cite{BrysonHo69}, \cite{locatelli2001optimal}, \cite{athans2006optimal}) were referenced for the derivations used in this thesis. In order to test the nonlinear optimization, numerical algorithms for the shooting method (\cite{richard1988douglas}, \cite{rao2009engineering}, \cite{keller1992numerical}) and the finite difference method (\cite{rao2009engineering}, \cite{keller1992numerical}) were used.

The Proportional-Integral-Derivative (PID) controller is a control loop feedback mechanism widely used to drive a system to a desired set point. The mechanism uses an error value as the input to the controller. PID controllers are common in industrial applications \cite{o2006reducing}. In the absence of the knowledge of an underlying process, the PID is considered the best method of control. It must be noted that PID controllers do not necessarily result in optimal control of the system. However, it is possible to achieve a desired system response by adjusting the mathematical parameters of the control expressions. This process is called ``tuning''. The tuning must satisfy many criteria within the limitations of PID control and the system itself. There are various tuning techniques \cite{bequette2003process}. For instance, there are the Ziegler-Nichols, manual tuning, and software tuning methods which can be applied to other control problems \cite{AngChong}, \cite{Bennett}.


\section{Organization of the Thesis}

The objective of this thesis is to develop a path-energy optimization technique that can operate on a near real-time schedule. Two different methods are discussed. We compare a classical optimal control technique with a simpler heuristic approach involving PID controller tuning. The organization of the chapters is as follows.

In Chapter 2, we present a detailed problem statement where the goal of the research project is defined. Chapter 3 uses the Euler-Lagrange equations of motion to derive a nonlinear dynamical model for the quad-rotor UAV. This mathematical model is the basis of the development of the control and energy optimization algorithms. In chapter 4, we define the various optimality conditions. Then we and formulate a generalized, classical optimal control scheme. Then we solve the boundary value problem generated by two methods and discuss their pros and cons. In Chapter 5, the classical optimal control scheme developed in the previous chapter is applied to the quad-rotor UAV. The resulting boundary value problem and its solution method is discussed. Chapter 6 deals with the PID/PD control technique. The control expressions are derived, and the method is tested. Results from these tests are discussed. Chapter 7 outlines a heuristic approach to the path-energy optimization problem and presents the simulation results of the control algorithm developed. Chapter 8 summarizes the results and proposes avenues for continued research.












