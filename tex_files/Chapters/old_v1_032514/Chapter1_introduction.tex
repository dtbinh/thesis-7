% Chapter 1

\chapter{Introduction} % Main chapter title

\label{Introduction} % For referencing the chapter elsewhere, use \ref{Chapter1} 

\lhead{Chapter 1. \emph{Introduction}} % This is for the header on each page - perhaps a shortened title

%-------------------------------------------------------------- --------------------------

The technology surrounding unmanned aerial vehicles (UAVs) and in particular, quad-rotor devices has seen tremendous development in recent years. Likewise, the creative application of this technology has expanded into many contexts. Like many new technologies, the early development of UAVs was mostly in a military context. This is not the case any more. The private sector has taken a huge interest in this technology. Also, the FAA has plans to produce concrete policy regarding the regulation of commercial applications of UAVs by 2015. This will allow for the rapid growth of a multi-billion dollar industry. UAV Technology is forcasted to be as pervasive and transformative as personal computers or cars. There are many applications for this technology which have the potential to save lives and collect scientific data that could inform state and federal legislation. Certainly, the range of potential applications will be further diversified as the technology sees more development. Specific advances in research in this area are described in the next two sections.


\section{Motivation}

One pervasive engineering problem that still exists in general with UAV technology is that of managing the energy usage. Quad-rotors specifically are plagued by very high energy demand. Unlike fixed wing UAVs which have the ability to soar or glide, multi-rotor UAV systems are entirely thrust driven. It is the natural instability of multi-rotor UAVs which make them extremely maneuverable, but this comes at the cost of continuous energy expenditure.

There has been very interesting work published on the energy optimization and trajectory planning of fixed wing UAVs. Given the ability to soar and utilize thermal gradients in the atmosphere, it is suggested that fixed wing UAVs have the potential to stay aloft permanently. Published works which address this problem are \cite{langelaan2007long}, \cite{klesh2009solar}, and \cite{lawrance2009guidance}. Since fixed wing and multi-rotor UAVs are so fundamentally different in their physical operation, procedures for managing their respective energy usage are also necessarily unique. Given this, previous works which focus on fixed wing UAVs can provide little more than inspiration for this thesis.



The use of PID controllers in industrial applications 
\cite{o2006reducing}

{\color{red}Are there any other related works ? what else can I put here??? }

Based on the literature searches conducted for this thesis, there has been no published work to date which approaches the problem of a simulation based energy optimization involving a quad-rotor dynamic model. {\color{green} But there are other problems that deal with energy optimization, even though it is not a quad-rotor model. This literature survey need not be restricted to the exact problem at hand, but the overall methods used to solve your problem}






\section{Prior Work}

In this section, we describe our work in relation to the body of published work that we intend to build upon.


In academic contexts, many advances in UAV and specifically quad-rotor research have provided the seeds for growth for this industry. The problem of basic stability and attitude control is solved in \cite{erginer2007modeling}, \cite{bouabdallah2004pid}, and \cite{Luukkonen}. In the

position and trajectory control

control using camera feedback

swarm control / distributed control

modeling wind forces


Excellent references for the background material for understanding the dynamical model of the quad-rotor as given by the Euler Lagrange formulation are \cite{marion1995classical} and  \cite{cornelius1970variational}. These references provided detailed derivations and discussion of the Euler - Lagrange equations of motion as well as related topics like Hamiltonian mechanics and the calculus of variations. Also, \cite{cornelius1970variational} provides an in depth review of the historical context surrounding the development of classical mechanics. 

The derivation of the quad-rotor dynamical model as well as attitude and position control via PD or PID controllers is discussed in \cite{erginer2007modeling}, \cite{bouabdallah2004pid}, \cite{Luukkonen}. These papers provide derivations of both the Euler - Lagrange and Newtonian formulations for the quad-rotor.


Several books on optimal control were used as reverences for the derivations in Chapter 3 and 4 : \cite{lewis2012optimal} , \cite{BrysonHo69}, \cite{locatelli2001optimal}, \cite{athans2006optimal}.

Various numerical techniques were found in \cite{richard1988douglas}, \cite{rao2009engineering},and \cite{keller1992numerical}.

{\color{red} need to be specific about the details of each reference, what are their conclusions/results , how does this work expand on that...} 


\section{Organization of the Thesis}

This work is to develop an energy optimization technique that can operate on a near real-time schedule. Two different methods are discussed. We compare a classical optimal control technique with a simpler heuristic approach involving PID controller tuning. 

Chapter 2 presents a detailed problem statement where the goal of the research is defined. In Chapter 3, a nonlinear dynamical model for the quad-rotor is developed. This mathematical model is the basis of the development of the control and energy optimization algorithms. {\color{red}In Chapter 4,...}


{\color{red}    +++++++++++++++++++++}

