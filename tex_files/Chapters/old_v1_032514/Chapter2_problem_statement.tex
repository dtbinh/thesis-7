% Chapter Template

\chapter{Problem Statement} % Main chapter title

\label{Chapter2} % Change X to a consecutive number; for referencing this chapter elsewhere, use \ref{ChapterX}

\lhead{Chapter 2. \emph{Problem Statement}} % Change X to a consecutive number; this is for the header on each page - perhaps a shortened title



We wish to find a set of control expressions for a quad-rotor UAV which minimizes the energy expended in flying between two known 3D points. In order to maintain focus on a tractable problem, some mathematical assumptions are made about the scenario. First, we assume that the flight path that will be optimized is free of obstacles. Second, we assume only modeled environmental variables. We use a mathematical model of the system derived from a Euler-Lagrange formulation as in \cite{Luukkonen} and \cite{bouabdallah2004pid}. 

In the classical optimal control approach,\cite{BrysonHo69} \cite{kirk70} {\color{red}WHY IS THIS REFERENCE BROKEN???} \cite{lewis2012optimal}, the control of the system and the optimization are represented in a single mathematical formulation. Solving the optimal control problem is achieved by solving a boundary value problem. For a highly non-linear system such as a quad-rotor, this becomes extremely involved. The classical optimal control approach is shown to be too computationally intensive for a real-time implementation because the result is a monstrous two point boundary value problem. Solving the theoretical optimal control problem would likely produce accurate results, but only after an inordinate lapse of time for computation. Also, the convergence of the solution is shown to be intermittent. 

 For our heuristic approach, full control of the UAV is attained by using PD attitude controllers in conjunction with PID controllers for position. This provides a platform for simulating the UAV as it flies from a known initial position to a desired set point location. The optimization procedure evaluates the results of these simulations for optimality as a function of the PID gains used in the position control expressions.   

It is pertinent to define what is meant by near real-time in our somewhat sterile mathematical context. We assume that the set of initial and final locations of the quad-rotor are defined by a user on a human time scale. Imagine a graphical user interface in which the desired location of the UAV is programmed. The quad-rotor then physically traverses the optimal path without more than a second or two of computation before the flight begins. For an autonomous UAV, the on board computational resources define an upper limit to the computational complexity of the control algorithm. It is our aim to design an energy optimized control scheme which meets these constraints. 

